\chapter{Introduction}
\label{sec:intro}

\kant[4] % Dummy text

\begin{equation}
    \iint_D \diff x \diff y
    =
    \int_0^{2\pi} \int_0^t \rho \diff \rho \diff t
    =
    \frac{4}{3} \pi^3.
\end{equation}

\section{Figures and Tables}

% Standalone with \input:
\begin{figure}[htbp]
    \centering
    \input{figures/ball}
    \caption[One ball]{One ball.}
\end{figure}

% Standalone with \includegraphics:
\begin{figure}[thbp]
    \centering
    \includegraphics{balls}
    \caption[Two balls]{Two balls.}
\end{figure}

% Todonotes:
\begin{figure}[hbp]
    \centering
    \missingfigure{Three balls.}
    \caption[Three balls]{Three balls.}
\end{figure}

\kant[5-6] % Dummy text

% Booktabs:
\begin{table}[htbp]
    \centering
    \begin{tabular}{@{}ll@{}}
        \toprule
        \textbf{Correct}               & \textbf{Incorrect}      \\
        \midrule
        \( \varphi \colon X \to Y \)   & \( \varphi : X \to Y \) \\[0.5ex]
        \( \varphi(x) \coloneqq x^2 \) & \( \varphi(x) := x^2 \) \\
        \bottomrule
    \end{tabular}
    \caption[Colons]{Proper colon usage.}
\end{table}

\begin{table}[htbp]
    \centering
    \begin{tabular}{@{}ll@{}}
        \toprule
        \textbf{Correct}     & \textbf{Incorrect}         \\
        \midrule
        \( A \implies B \)   & \( A \Rightarrow B \)      \\
        \( A \impliedby B \) & \( A \Leftarrow B \)       \\
        \( A \iff B \)       & \( A \Leftrightarrow B \)  \\
        \bottomrule
    \end{tabular}
    \caption[Arrows]{Proper arrow usage.}
\end{table}

% Tablefootnote and multirow:
\begin{table}[htbp]
    \centering
    \begin{tabular}{@{}ll@{}}
        \toprule
        \textbf{Correct}
        & 
        \textbf{Incorrect}
        \\
        \midrule
        \( -1 \) 
        & 
        -1
        \\[0.3ex]
        1--10
        &
        1-10
        \\[0.3ex]
        Birch--Swinnerton-Dyer\tablefootnote{It is now easy to tell that Birch and Swinnerton-Dyer are two people.} conjecture
        &
        Birch-Swinnerton-Dyer conjecture
        \\[0.3ex]
        The ball \dash which is blue \dash is round.
        &
        \multirow{ 2}{*}{The ball - which is blue - is round.}
        \\[0.3ex]
        The ball---which is blue---is round. 
        &
        \\
        \bottomrule
    \end{tabular}
    \caption[Dashes]{Proper dash usage.}
\end{table}

\begin{table}[hbtp]
    \centering
    \begin{tabular}{@{}*{2}{p{0.5\textwidth}}@{}}
        \toprule
        \textbf{Correct} &  \textbf{Incorrect}
        \\
        \midrule
        \enquote{This is an \enquote{inner quote} inside an outer quote}
        &
        'This is an "inner quote" inside an outer quote'
        \\
        \bottomrule
    \end{tabular}
    \caption[Quotation marks]
    {Proper quotation mark usage.
    The \texttt{\textbackslash enquote} command chooses the correct
    quotation marks for the specified language.}
\end{table}

\section{Outline}

The rest of the text is organised as follows:
\begin{description}
    \item[\cref{sec:second}]
    is second to none, with the notable exception of \cref{sec:intro}.
    The main tool introduced here is the employment of unintelligible sentences.

    \item[\cref{sec:third}]
    asserts the basic properties of being the third chapter of a text.
    This section reveals the shocking truth of filler content.

    \item[\cref{sec:fourth}]
    demonstrates how easily one can get to four chapters by simply using the \texttt{kantlipsum} package to generate dummy words.

    \item[\cref{sec:first-app}]
    features additional material for the specially interested.

    \item[\cref{sec:second-app}]
    consists of results best relegated to the back of the document,
    ensuring that nobody will ever read it.
\end{description}